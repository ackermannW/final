\thispagestyle{plain}
\begin{center}
    \Large
    \textbf{Principi vođenja i upravljanja taktičkih projektila kratkog dometa}
    
    \vspace{0.4cm}
        
    \vspace{0.4cm}
    \textbf{Mirza Hodžić}
    
    \vspace{0.9cm}
    \textbf{Sažetak}
\end{center}
Većina savremenih taktičkih projektila su samonavođeni projektili (fire and forget). Za
osiguranje visoke performanse neophodan je kvalitetan sistem vođenja (guidance) i
upravljanja (control) projektila. Kretanjem projektila pod dejstvom pogonskih (raketni
motor), reaktivnih i upravljačkih (aerodinamičke) sila mijenjaju se njegovi prostorni
kinematski odnosi u odnosu na cilj. Sistem vođenja registruje te promjene i u odnosu na dati
zadatak i usvojeni zakon viđenja formira grešku vođenja. Tačnost sistema vođenja zavisi od
senzorskog sistema i sistema za praćenje cilja (target tracker, target seeker). Signal greške
vođenja predstavlja referentnu ulaznu vrijednost za sistem za upravljanje kretanjem projektila
(autopilot) koji nastoji da smanji grešku vođenja. Tačnost autopilota zavisi od primijenjenih
senzora kretanja i zakona upravljanja.