\documentclass[12pt, twoside]{report}
\usepackage{sectsty}
\usepackage[utf8]{inputenc}
\usepackage[croatian]{babel}
\usepackage{graphicx}

\graphicspath{ {Images/} }
%\usepackage{lipsum}
\usepackage[a4paper,width=150mm,top=25mm,bottom=25mm,bindingoffset=6mm]{geometry}
\usepackage{fancyhdr}
\usepackage{xcolor}
\usepackage{tcolorbox}
%\usepackage{draftwatermark}
\usepackage{amsmath,tabstackengine}
\usepackage[section]{placeins}
\usepackage{tikz}
\usepackage{textcomp}
\usepackage[OT1]{fontenc}
\usepackage[utf8]{inputenc}
\usepackage{lmodern}
\usepackage{csquotes}
\usepackage[framed,numbered,autolinebreaks,useliterate]{mcode}
\usetikzlibrary{shapes,arrows,positioning,calc}


\pagestyle{fancy}
\fancyhead{}
\fancyhead[RO,LE]{Principi vođenja i upravljanja taktičkih projektila kratkog dometa}
\setlength{\headheight}{15pt}
\fancyfoot{}
\fancyfoot[CO,RE]{\thepage}
\renewcommand{\headrulewidth}{0.4pt}
\renewcommand{\footrulewidth}{0.4pt}


%\SetWatermarkText{Nedovršeno}
%\SetWatermarkScale{1}
% https://en.wikibooks.org/wiki/LaTeX/Floats,_Figures_and_Captions
\usepackage{caption}
\usepackage{subcaption}

% References
\usepackage[style=authoryear,sorting=ynt]{biblatex}
\addbibresource{references.bib}

% https://en.wikibooks.org/wiki/LaTeX/Text_Formatting#Line_Spacing
\usepackage{setspace}
\singlespacing
\onehalfspacing
%\doublespacing
%\setstretch{1.1}
\hbadness=99999
\usepackage{titlesec,blindtext,color}

\definecolor{gray75}{gray}{0.75}
\newcommand{\hsp}{\hspace{20pt}}
\titleformat{\chapter}[hang]{\Huge\bfseries}{}{0pt}{\Huge\bfseries}
\begin{document}
\tikzset{
block/.style = {draw, fill=white, rectangle, minimum height=3em, minimum width=3em},
tmp/.style  = {coordinate}, 
sum/.style= {draw, fill=white, circle, node distance=1cm},
input/.style = {coordinate},
output/.style= {coordinate},
pinstyle/.style = {pin edge={to-,thin,black}
}
}
\begin{titlepage}
    \begin{center}
        \vspace*{1cm}
        
        \Huge
        \textbf{Navigacija i upravljanje projektila}
        
        %\vspace{0.5cm}
        
       % \LARGE
        %Thesis Subtitle
        
        \vspace{1.5cm}
        
        \Large
        \textbf{Mirza Hodžic}\\
        
        \vspace{0.5cm}
        
        Mentor: prof. dr. Naser Prljača\\
        
        \vfill
        
        \includegraphics[width=0.5\textwidth]{Images/preuzmi.png}
        
        \vspace{0.8cm}
        
        %\Large
        %A thesis presented for the degree of Doctor of Philosophy
        
        \vspace{0.5cm}
        
        \LARGE
	    \textsc{Fakultet eleketrotehnike\\
	    Univerzitet u Tuzli}
	    
	    \begin{flushright}
	
	    \Large
	%    \date{Oktobar, 2020}
	
	    \end{flushright}
        
    \end{center}
\end{titlepage}

\chapter*{Posveta}
Mojim roditeljima

%\chapter*{Declaration}
%I declare that..

%\chapter*{Acknowledgements}
%I want to thank...
\tableofcontents

%\listoffigures
%\listoftables
\newpage
\thispagestyle{plain}
\begin{center}
    \Large
    \textbf{Principi vođenja i upravljanja taktičkih projektila kratkog dometa}
    
    \vspace{0.4cm}
        
    \vspace{0.4cm}
    \textbf{Mirza Hodžić}
    
    \vspace{0.9cm}
    \textbf{Sažetak}
\end{center}
Većina savremenih taktičkih projektila su samonavođeni projektili (fire and forget). Za
osiguranje visoke performanse neophodan je kvalitetan sistem vođenja (guidance) i
upravljanja (control) projektila. Kretanjem projektila pod dejstvom pogonskih (raketni
motor), reaktivnih i upravljačkih (aerodinamičke) sila mijenjaju se njegovi prostorni
kinematski odnosi u odnosu na cilj. Sistem vođenja registruje te promjene i u odnosu na dati
zadatak i usvojeni zakon viđenja formira grešku vođenja. Tačnost sistema vođenja zavisi od
senzorskog sistema i sistema za praćenje cilja (target tracker, target seeker). Signal greške
vođenja predstavlja referentnu ulaznu vrijednost za sistem za upravljanje kretanjem projektila
(autopilot) koji nastoji da smanji grešku vođenja. Tačnost autopilota zavisi od primijenjenih
senzora kretanja i zakona upravljanja.

%\chapter[]{Uvod}
%\input{Chapters/introduction}

\chapter{Jednačine kretanja tijela}
\section{Koordinatni sistemi}
Orijentacija osa koordinatnog sistema preko kojih su određeni 
vektori ili tenzori potpuno je proizvoljna. Obično se jedna od osi(e.g. $x$ osa) poravnava 
sa geometrijskom osom tijela. 
Ako se tijelo kreće stalnom brzinom tada se jedan koordinatni sistem može 
koristiti za sve veličine, međutim ako se tijelo rotira tada se naslućuju dva koordinatna sistema:
\begin{itemize}
    \item Koordinatni sistem vezan za zemlju
    \item Koordinatni sistem vezan za tijelo
\end{itemize}
Koordinatni sistem vezan za zemlju je inercijalni iako se zemlje rotira u odnosu na geomtrijsku osu.
Sastoji se od tri ordinate, jedna predstavlja poziciju po sjevernoj osi, jedna po lokalnoj istočnoj osi 
i jedna predstavlja vertikalnu poziciju. Ose koordinatnog sistema vezanog za zemlju su 
označene sa $X_e, Y_e, Z_e$. Drugim riječima, $X_e$ i $Y_e$ leže u ravni dok je $Z_e$ usmjeren ka centru Zemlje.\\
Koordinatni sistem vezan za tijelo sastoji se iz tri ordinate sa ishodištem u centru gravitacije letjelice: $x$ osa koja je
usmjerena ka nosu letjelice tj. podudara se sa longitudinalnom osom, $y$ ose koja je usmjerena ka desnom krilu letjelice i $z$ ose koja dopunjava lijevo orijentisani 
koordinatni sistem.
\begin{figure}[!ht]
    \centering
    \includegraphics[scale=0.7]{body.png}
    \caption{Koordinatni sistem vezan za tijelo}
    \label{fig:KBS}
\end{figure}
Da se definiše položaj letjelice u odnosu na koordinatni sistem koriste se Eulerovi uglovi($\psi, \theta, \phi$).
\begin{figure}[!ht]
    \centering
    \includegraphics[scale=0.7]{earth-body.JPG}
    \caption{Eulerovi uglovi}
\end{figure}
Ovo znači da se bilo koja rotacija, odnosno transformacija iz sistema tijela u sistem Zemlje može postići sa tri rotacije oko osi i to prva 
rotacija za ugao $\phi$ oko longitudinalne, za ugao $\theta$ oko lateralne i za ugao 
$\psi$ oko normalne ose. Transformacija $T_b^e$ koja ostvaruje transformaciju iz 
koordinatnog sistema vezanog za zemlju u koordinatni sistem vezan za tijelo je data sa:
\begin{equation}
    T_{b}^z = T_1(\phi)T_2(\theta)T_3(\psi)
\end{equation}
,gdje je:
\begin{align}
    T_1(\phi) =& \begin{bmatrix}
        1 & 0 & 0\\
        0 & \cos\phi & \sin\phi \\
        0 & -\sin\phi & \cos\phi 
    \end{bmatrix}\\
     T_2(\theta)=& \begin{bmatrix}
        \cos\theta & 0 & -\sin\theta \\
         0 & 1 & 0\\
         \sin\theta & 0 & \cos\theta
    \end{bmatrix}\\
    T_3(\psi) = &\begin{bmatrix}
        \cos\psi & \sin\phi & 0\\
        -\sin\phi & \cos\phi & 0\\
        0 & 0 & 1
    \end{bmatrix}
\end{align}
,odnosno:
\begin{equation}
    {\footnotesize
    T_{b}^z=\begin{bmatrix}
        cos\theta cos\psi && cos\theta sin\psi && -sin\theta\\
        sin\phi sin\theta cos\psi-cos\phi sin\psi && sin\phi sin\theta sin\psi +cos\phi cos\psi&& sin\phi cos\theta\\
        c\phi s\theta c\psi+sin\phi sin\psi && cos\phi sin\theta sin\psi -sin\phi cos\psi&& cos\phi cos\theta\\
    \end{bmatrix}}
    \label{eq:ztob}
\end{equation}
Treba primjetiti da rezultantna matrica $T_{z \to b}$ može imati singularitete, pa se domen
Eulerovih uglova ograničava na sljedeći način:
\begin{align*}
    -\pi \leq \phi <\pi \quad ili \quad 0\leq\phi<2\pi \\
    -\pi \leq \psi <\pi \qquad \qquad \qquad \qquad \\
    -\frac{\pi}{2}\leq \theta \leq \frac{\pi}{2} \quad ili \quad 0\leq\psi<2\pi
\end{align*}
Ovo znači da u ovom slučaju postoji beskonačno mnogo načina da se ostvari željena transformacija.
Ovaj problem se može riješiti uvođenjem jediničnog kvaterniona.
Još jedan iznimno važan koordinatni sistem je \textit{koordinatni sistem brzine tijela(BKS)}. Ovaj Koordinatni
sistem se koristi kad god relativno kretanje objekta u odnosu na okolinu ima za posljedicu pojavu 
reaktivnih sila. Koordinatni sistem brzine je vezan za vektor brzine objekta. Ishodište kooridnatnog sistema 
sitema brzine tijela se podudara sa centrom mase tijela(centar mase se može mijenjati tokom leta zbog utroška goriva), dok 
je $X$ osa kolinearna sa vektorom brzine objekta. Druge dvije ose se proizvoljno definišu u ravni 
normalnoj na vektor brzine. Najčešće se uzima da $Z$ osa zadovoljava barem jedan od naredna dva uslova:
\begin{itemize}
    \item $Z$ leži u presjeku ravni normalne na vektor brzine i vertikalne ravni simetrije pokretnog objekta.
    \item $Z$ leži u presjeku ravni normalne na vektor brzine i vertikalne ravni referentnog koordinatnog sistema.
\end{itemize}
Koordinatni sistem brzine je prikazan na slici \ref{fig:vks}.
\begin{figure}[!ht]
    \centering
    \includegraphics[scale=0.5]{vks.PNG}
    \caption{Koordinatni sistem vezan za vektor brzine}
    \label{fig:vks}
\end{figure}
Ugao između $X$ ose sistema tijela i $X$ ose sitema brzine je označen sa $\eta$ i ovim uglom se i definiše 
koordinatni sistem brzine tijela. Ugao $\alpha$ je ugao ugao između $Z_b$ ose  sistema tijela i projekcije 
vektora brzine na vertikalnu ravan sistema tijela. Ovaj ugao se zove napadni ugao o kojem će više riječi biti kasnije. 
Ugao $\beta$ je ugao između vektora brzine i vertikalne ravni sistema tijela. Ovaj ugao se zove ugao klizanja i o njemu će 
više riječi biti kasnije. Transformacija sistema brzine u sistem tijela se postiže rotacijom 
oko $Y$ ose sistema tijela za ugao $\alpha$ praćene rotacijom oko $Z$ ose dobijenog sistema za ugao $\beta$. 
Odgovarajuća matrica transformacije je:
\begin{equation}
    T_{v}^b = \begin{bmatrix}
        \cos\alpha & 0 & -\sin\alpha \\
        0& 1& 0\\
        \sin\alpha & 0 & \cos\alpha
    \end{bmatrix}
    \begin{bmatrix}
        \cos\beta & \sin\beta & 0\\
        -\sin\beta & \cos\beta & 0\\
        0 & 0& 1
    \end{bmatrix}
\end{equation}
Nakon množenja matrica, dobija se:
\begin{equation}
    T_{v}^z = \begin{bmatrix}
        \cos\alpha\cos\beta & \cos\alpha\sin\beta & -\sin\alpha \\
        -\sin\beta & \cos\beta & 0\\
        \sin\alpha\cos\beta & 0 & \cos\alpha
    \end{bmatrix}
    \label{eq:VtoB}
\end{equation}
Veza između sistema Zemlje(inercijalnog sistema) i sistema brzine je data uglovima $\Theta$(ugla elevacije vektora brzine)
i $\Psi$, ugla azimuta vektora brzine. Transformacija iz inercijalnog sistema u sistem brzine se dobija rotacijom 
za $\Theta$ oko $X$ ose sistema brzine, zatim rotacijom oko $Z$ ose za $\Psi$. Odgovarajuća matrica transformacije je:
\begin{equation}
    T_{z}^v = \begin{bmatrix}
        \cos\Theta & 0 & -\sin\Theta \\
        0& 1& 0\\
        \sin\Theta & 0 & \cos\Theta
    \end{bmatrix}
    \begin{bmatrix}
        \cos\Psi & \sin\Psi & 0\\
        -\sin\Psi & \cos\Psi & 0\\
        0 & 0& 1
        \end{bmatrix}
        \label{eq:ztov}
\end{equation}
Nakon množenja matrica, dobija se:
\begin{equation}
    T_{z}^v = \begin{bmatrix}
        \cos\Theta\cos\Psi & \cos\Theta\sin\Psi & -\sin\Theta \\
        -\sin\Psi & \cos\Psi & 0\\
        \sin\Theta\cos\Psi & 0 & \cos\Theta
    \end{bmatrix}
\end{equation}
Sada su dobijene matrice koje opisuju transformacije iz sistema Zemlje u sistem tijela, 
iz brzinskog sistema u sistem tijela i iz sistema Zemlje u brzinski koordinatni sistem. 
Ako je potrebna obrnuta transformacija, može se iskoristiti osobina da elementarne matrice transformacije 
imaju ortogonalne kolone, tj. njihov skalarni proizvod je nula. Odavde slijedi 
da je transponovana matrica elementarne transformacije jednaka svom inverzu, odnosno $T_i(\epsilon)^T = T_i^{-1}(\epsilon)$.
Uzmimo sada $T_{z}^b$. Vrijedi:
\begin{equation*}
    T_{z}^b = T_1(\phi)T_2(\theta)T_3(\psi)
\end{equation*}
pa je:
\begin{align*}    
    { T_{z}^b}^T &= T_3^T(\psi)T_2^T(\theta)T_1^T(\phi)\\ & =  T_3^{-1}(\psi)T_2^{-1}(\theta)T_1^{-1}(\phi)\\
    & = [T_1(\phi)T_2(\theta)T_3(\psi)]^{-1} = {T_{z}^b}^{-1} = T_{b}^z
\end{align*}
Ovo sada znači da se inverzna matrica transformacije može naći transponovanjem originalne matrice 
transformacije. 
\section{Jednačine kretanja čvrstog tijela}
Sada ćemo posmatrati tipični projektil i izvesti jednačine koje opisuju njegovo kretnaje.
Pretpostaviti će se da čvrsto tijelo nema promjena u obliku pri kretanju. Translacija tijela 
podrazumijeva da svaka duž koja spaja bilo koje dvije tačke u tijelu bude paralelna svojoj
datoj originalnoj poziciji, prema tome čvrsto tijelo se može posmatrati kao čestica čija je 
masa skoncentrisana u jednoj tački koja se zove \textit{centar mase}. Dalje se pretpostavlja 
da se oblik tijela ne mjenja usljed djelovanja sila na tijelo. Ovom pretpostavkom se 
dobija da je međusobni utjecaj različitih dijelića tijela eleiminisan pa se transalcija može potpuno opisati
translacijom centra mase i da se rotacija može potpuno opisati rotacijom oko centra mase. Dodatno 
pretpostavlja se da se ravan simetrije poklapa sa ravninom $X_b - Z_b$ kao što je to prikazano na 
slici \ref{fig:KBS}. Također pretpostavlja se da je masa tijela konstantna. Važno je 
napomenuti da se jednačine tijela određuju u koordinatnom sistemu vezanom za tijelo. 
Nadalje, projektil ima šest stepeni slobode(6-DOF). Ovih šest stepeni se sastoje iz od tri translacije i 
tri rotacije. Translacije se sastoje od kretanja duž osi $X_b,Y_b,Z_b$ brzinom $v_m=(u,v,w)$, a rotacije se sastoje 
od rotacija oko ovih osi ugaonom brzinom $\omega = (P,Q,R)$. Šest stepeni slobode je prikazano
na slici \ref{fig:dof} 
\begin{figure}
    \centering
    \includegraphics[scale=0.6]{6dof.JPG}
    \caption{Predstava šest stepeni slobode}
    \label{fig:dof}
\end{figure}
Kao što je ranije rečeno dinamički model projektila se dobija Newtonovim zakonom dinamike,
koji kaže da je suma svih vanjskih sila jednaka brzini promjene impulsa tijela i da je 
suma svih vanjskih momenata jednaka brzini promjene momenta impulsa. Prema tome vrijede relacije:
\begin{equation}
    \sum F=\frac{d(mv_m)}{dt}|_{Zemlja}
    \label{eq:f}
\end{equation}
\begin{equation}
    \sum M=\frac{dH}{dt}|_{Zemlja}
    \label{eq:m}
\end{equation}
gdje je $H$ ugaoni momentum a $\sum M$ je suma svih vanjskih momenata koji djeluju na tijelo. Naravno, prethodne 
relacije predstavljaju promjene vektora u odnosu na inercijalni prostor. Rezultantna vanjska sila koja 
djeluje na tijelo se može razložiti na sile koje djeluju po osama koordinatnog sistema 
vezanog za tijelo projektila, pa se može napisati:
\begin{equation}
    \sum \Delta F=\sum \Delta  F_xi+\sum \Delta  F_yj+\sum \Delta  F_zk
    \label{eq:sum}
\end{equation}
Poredeći prethodnu jednačinu sa \ref{eq:f} dobija se:
\begin{equation}
    F_x=\frac{d(mu)}{dt}, F_y=\frac{d(mv)}{dt}, F_z=\frac{d(mw)}{dt}
    \label{eq:31}
\end{equation}
Analogno, dobija se da vrijedi:
\begin{equation}
    L=\frac{dH_x}{dt},M=\frac{dH_y}{dt},N=\frac{dH_z}{dt}
    \label{eq:32}
\end{equation}
Gdje su $L,M$ i $N$ moment valjanja, moment propinjanja i moment zakretanja respektivno i 
$H_x, H_y$ i $H_z$ su komponente momenta impulsa duž osa tijela. 
Sada želimo proširiti jednačine \ref{eq:31} i \ref{eq:32} kako bismo dobili 
jednačine kretanja za svaki stepen slobode. U svrhu toga koristi se formula za 
brzinu promjenu brzine projektila u inercijalnom sistemu, tj. u koordinatnom sistemu 
vezanom za zemlju i ona je data relacijom:
\begin{equation}
    \left( \frac{dv_m}{dt}\right)_{Zemlja}=\left(\frac{dv_m}{dt}\right)_{tijelo}+\omega \times v_m
\end{equation}
Prema tome vrijedi da je ukupna vanjska sila koja djeluje na tijelo data sa:
\begin{equation}
    F=m\left(\frac{dv_m}{dt}\right)_{tijelo}+m(\omega \times v_m)
    \label{eq:force}
\end{equation}
gdje je vektorski proizvod linearne brzine i ugaone brzine dat sa:
\begin{equation}
    \omega \times v_m=\begin{vmatrix}
        i&j&k\\
        P&Q&R\\
        u&v&w\\
    \end{vmatrix}=(wQ-vR)i+(uR-wP)j+(vP-uQ)k
\end{equation}
Koristeći se činjenicom da je $v_m=ui+vj+wk$ i uvrštavanjem prethodne jednačine u \ref{eq:force} dobija se:
\begin{equation}
    \sum \Delta F=m(\dot{u}i+\dot{v}j+\dot{w}k)+(wQ-vR)i+(uR-wP)j+(vP-uQ)k
\end{equation}
Sada, poredeći sa \ref{eq:sum} dobijaju se jednačine:
\begin{equation}
    \sum \Delta F_x=m(\dot{u}+wQ-vR)
    \label{eq:r1}
\end{equation}
\begin{equation}
    \sum \Delta F_y=m(\dot{v}+uR-wP)
    \label{eq:r2}
\end{equation}
\begin{equation}
    \sum \Delta F_z=m(\dot{w}+vP-uQ)
    \label{eq:r3}
\end{equation}
Prethodno dobivene tri jednačine predstavljaju \textit{linearne jednačine kretanja}. Sada treba odrediti 
ove tri jednačine za rotaciono kretanje. Da bi se to postiglo potrebno je imati izraz za 
moment impulsa $H$ kao što imamo izraz za impuls kod translatornog kretanja. Moment impulsa oko 
proizvoljne tačke $O$ materijalne tačke je dat sa:
\begin{equation}
    H=r\times mV=mr\times (\omega \times r)
\end{equation}
Vektor momenta impulsa $H$ je normalan $r$ i na $v$ i $H$ je usmjeren isto kao i moment impulsa $M$.
Moment impulsa cijelog tijela oko tačke $O$ je dat sa:
\begin{equation}
    H=\sum r\times mv_m=\sum mr\times (\omega \times r)=\sum m\left[\omega(r\cdot r )-r(r\cdot \omega) \right]
\end{equation}
ili u formi integrala:
\begin{equation}
    H=\int r\times (\omega \times r)dm
\end{equation}
Sada slijedi:
\begin{equation}
    \omega \times r=\begin{vmatrix}
        i&j&k\\
        P&Q&R\\
        x&y&z\\
    \end{vmatrix}=(zQ-yR)i+(xR-zP)j+(yP-xQ)k
\end{equation}
 i konačno:
 \begin{equation}
    r\times (\omega \times r)=\begin{vmatrix}
        i&j&k\\
        x&y&z\\
        zQ-yR&xR-zP&yP-xQ\\
    \end{vmatrix}
 \end{equation}
 Sada se konačno dobija izraz za moment impulsa:
 \begin{equation}
    \begin{split}      
     H&=i\int \left[ (y^2+z^2)P -xyQ -xzR \right]dm+j\int\left[ (z^2+x^2)Q-yzR-xyP \right]dm\\ 
     &+k\int \left[ (x^2+y^2)R-xzP-yzQ \right]dm    
    \end{split}
    \end{equation}

    Kada se uvedu oznake:
    \begin{equation}
        \begin{split}           
        I_x&=\int (y^2+z^2)dm, I_z=\int (y^2+x^2)dm,I_z=\int (x^2+z^2)dm\\
        &I_{xy}=\int xydm, I_{yz}=\int yzdm,I_{xz}=\int xzdm 
    \end{split}
    \end{equation}
Tada se dobija:
\begin{equation}
    H=(PI_x-RI_{xz})i+QI_yj+(RI_z-PI_{xz})k
\end{equation}
Sada se vektor momenta impulsa može zapisati preko svojih komponenti:
\begin{equation}
    H_x=PI_x-RI_{xz}
\end{equation}
\begin{equation}
    H_y=QI_y
\end{equation}
\begin{equation}
    H_z=RI_z-PI_{xz}
\end{equation}
Sada su potrebni izvodi momenta impulsa kako bismo dobili izraz za rezultantni moment.
Pošto je izvod vektora u inercijalnom prostoru jednak zbiru izvoda pojedinačnih komponenti vektora. Prema tome 
vrijedi:
\begin{equation}
    \frac{dH_x}{dt}=I_x\frac{dP}{dt}-I{xz}\frac{dR}{dt}
\end{equation}
\begin{equation}
    \frac{dH_y}{dt}=I_y\frac{dQ}{dt}
\end{equation}
\begin{equation}
    \frac{dH_z}{dt}=I_z\frac{dR}{dt}-I_{xz}\frac{dP}{dt}
\end{equation}
Relacija \ref{eq:m} se može napisati kao:
\begin{equation}
    \sum \Delta M=\frac{dH}{dt}+\omega \times H
\end{equation}
Ako se uvaži da je $\sum \Delta M=\sum \Delta Li + \sum \Delta Mj+\sum \Delta Nk$, korištenjem prethodno dobivenih 
izraza za izvod momenta impulsa dobija se:
\begin{equation}
    \sum \Delta L=\dot{P}I_x+QR(I_z-I_y)-(\dot{R}+PQ)I_{xz}
\end{equation}
\begin{equation}
    \sum \Delta M=\dot{Q}I_y+PR(I_x-I_z)+(P^2-R^2)I_{xz}
\end{equation}
\begin{equation}
    \sum \Delta N=\dot{R}I_y+PQ(I_y-I_x)-(\dot{P}-QR)I_{xz}
\end{equation}
Prethodne tri jednačine zajedno sa jednačinama \ref{eq:r1},\ref{eq:r2} i \ref{eq:r3} predstavljaju
jednačine projektila sa šest stepeni slobode. Ove jednačine su simultane linearne jednačine 
kretanja sa šest promjenjivih $u,v,w,P,Q$ i $R$ koje potpuno opisuju kretanje 
čvrstog tijela. Rješenja ovih jednačina se mogu dobiti numeričkim metodama na digitalnom 
računaru. Analitička rješenja dovoljne tačnosti se mogu dobiti linearizacijom. $I_x,I_y$ i $I_{xz}$ su konstantne 
i za projektile sa krstastom konfiguracijom vrijedi $I_y=I_z$ i $I_{xz}$. Prema tome, vrijedi:
\begin{equation}
    \sum \Delta L=\dot{P}I_x+QR(I_z-I_y)
\end{equation}
\begin{equation}
    \sum \Delta M=\dot{Q}I_y+PR(I_x-I_z)
\end{equation}
\begin{equation}
    \sum \Delta N=\dot{R}I_z+PQ(I_y-I_x)
\end{equation}
Transformacijom prethodnih jednačina dobija se:
\begin{equation}
    \frac{dP}{dt}=QR\frac{I_y-I_z}{I_x}+\frac{L}{I_x}
    \label{eq:q1}
\end{equation}
\begin{equation}
    \frac{dQ}{dt}=PR\frac{I_z-I_x}{I_y}+\frac{M}{I_y}
    \label{eq:q2}
\end{equation}
\begin{equation}
    \frac{dR}{dt}=PQ\frac{I_x-I_y}{I_z}+\frac{N}{I_z}
    \label{eq:q3}
\end{equation}
Sada je još potrebno odrediti ugaone brzine u zavisnosti od Eulerovih uglova. Izvođenje ovih jednačina zahtjeva 
pronalaženje izvoda matrice transformacije, što je poprilično zahtjevno, pa će ovdje biit samo navedene 
diferencijalne jednačine koje daju brzinu promjene Eulerovih uglova:
\begin{equation}
    \frac{d\psi}{dt}=(Q\sin\phi +R\cos\phi)/\cos\theta
    \label{eq:w1}
\end{equation}
\begin{equation}
    \frac{d\theta}{dt}=Q\cos\phi-R\sin\phi
    \label{eq:w2}
\end{equation}
\begin{equation}
    \frac{d\phi}{dt}=P+\left( \frac{d\psi}{dt} \right)\sin\theta
    \label{eq:w3}
\end{equation}
Sada koristeći matricu transformacije $C_e^b$ se mogu dobiti komponente
brzine u koordinatnom sistemu Zemlje:
\begin{equation}
    \begin{bmatrix}
        \dot{X_z}\\
        \dot{Y_z}\\   
        \dot{Z_z}\\
    \end{bmatrix}=C_e^b\begin{bmatrix}
        u\\
        v\\   
        w\\
    \end{bmatrix}
    \label{eq:q}
\end{equation}
Sada je jasno da se integracijom jednačina \ref{eq:q1},\ref{eq:q2} i \ref{eq:q3} dobijaju 
ugaone brzine u sistemu tijela, a integracijom jednačina \ref{eq:w1},\ref{eq:w2} i \ref{eq:w3}  se dobija 
orijentacija u odnosu na zemlju. Da bi se dobila pozicija tijela u odnosu na sistem Zemlje
treba riješiti matričnu jednačinu \ref{eq:q}. Da bi se ona mogla numerički riješiti treba 
naći izraze za izvode brzina u sistemu tijela. Oni se mogu dobiti iz jednačina 
\ref{eq:r1}, \ref{eq:r2} i \ref{eq:r3}. Nakon transformacije ovih jednačina ima se:
\begin{equation}
    \frac{du}{dt}=vR-wQ+F_x/m
\end{equation}
\begin{equation}
    \frac{dv}{dt}=wP-uR+F_y/m
\end{equation}
\begin{equation}
    \frac{dw}{dt}=uQ-vR+F_z/m
\end{equation}
Sada se nakon rješavanja prethodne tri jendačine mogu dobiti vrijednosti brzina u sistemu tijela 
te nakon toga može se riješiti jednačina \ref{eq:q} i tako dobiti poziciju u odnosu na sistem Zemlje.
Prethodnih 12 jednačina se može predstaviti u prostoru stanja ako se uzme vektor varijabli stanja:
\[\vec{X}=\left[ u \quad v\quad w\quad P\quad Q\quad R\quad \phi \quad \theta \quad
 \psi \quad x_z\quad y_z\quad z_z\quad \right]^T\] i vektor upravljačkih 
promjenljivih:
\[\vec{u}=\left[\delta_v \quad \delta_P\quad \delta_e \right]^T\] 
,gdje je $\delta_v$ ugao otklona krmila visine, $\delta_P$, ugao otklona krmila 
i $\delta_e$, ugao otklona elerona. Upravljačke varijable se na prvu ruku ne vide u predstavljenim jednačinama, 
ali ubrzo ćemo se uvjeriti da sile i momenti koji djeluju na projektil zavise upravo od ovih upravljačkih 
varijabli.  
Ovime se dobija nelinearna vektorska jednačina:
\begin{equation}
    \dot{\vec{X}}=f(\vec{X},\vec{u},t)
\end{equation}
Prethodna jednačina je doista nelinearna najprije zbog prirode modela, postojanja trigonometrijskih funkcija i 
zbog nelinearne zavisnosti sila i momenata od otklona upravljačkih površina. Kako bi se izvršila sinteza 
regulatora prethodna jednačina se najprije treba linearizirati za određeni režim leta. Već se nadzire 
da se linearizacija može izvršiti nalaženjem prvih izvoda vekotrske funkcije $f(\vec{X},\vec{u},t)$ za određene uslove leta. 
Dobijena matrica bi imala 144 elementa koji su ustvari prvi izvodi raznih parametara modela pa je evidentno 
da treba poznavati zavisnosti parametra od vremena i međusobne zavisnosti varijabli stanja.  







\chapter{Sile koje djeluju na projektil}

Sile koje djeluju na projektil su u letu su aerodinamičke, pogonske sile i 
gravitaciona sila. Ove sile se mogu razložiti po osama kooridnatnog sistema vezanog 
za tijelo. 
\section{Aerodinamičke sile}
Aerodinamička sila je posljedica djelovanja pritiska okolnog fluida na tijelo u pokretu. 
Aerodinamička sila se može razložiti na tri komponente koje su definisane u nastavku:
\begin{itemize}
    \item \textbf{Uzgon}- Uzgon je komponenta rezultantne aerodinamičke sile 
    koja je normalna na relativno kretanje vjetra.
    \item \textbf{Otpor}- Otpor je komponenta rezultantne aerodinamičke sile 
    koja je paralelna relativnom kretanju vjetra.
    \item \textbf{Bočna sila}- Bočna sila je komponenta rezultantne aerodinamičke sile 
    koja je normalna na uzogn i otpor. 
\end{itemize}
Ovdje se posmatraju projektili koji se zakreću da bi skrenuli(skid to turn) i 
kod takvih projektila aerodinamičke sile su date sa:
\begin{equation}
   \text{Uzgon} \quad R_x=C_xqS
   \label{eq:a1}
\end{equation}
\begin{equation}
    \text{Otpor} \quad R_z=C_zqS
    \label{eq:a2}
\end{equation}
\begin{equation}
    \text{Bočna sila} \quad R_y=C_yqS
    \label{eq:a3}
\end{equation}
,gdje su $C_x,C_y$ i $C_z$ aerodinamički koeficijenti, $q$ dinamički pritisak slobodnog strujanaja
u tački daleko od objekta i iznosi $q=\frac{1}{2}\rho v^2$, $S$ je referentna površina i 
$v$ je brzina vazduha, $\rho$ predstavlja atmosferski pritisak.
\\
U opštem slučaju koeficijenti aerodinamičkih sila su funkcije varijabli stanja pa se može
napisati:
\begin{equation}
    C_x=C_x(\alpha ,\beta, M,q,\delta_v,\delta_P,\delta_e)
\end{equation}
,gdje je $M$ Mahov broj- odnos tekuće brzine i brzine zvuka, $\alpha$ napadni ugao i 
$\beta$ ugao klizanja. Slično tako vrijedi:
\begin{equation}
    C_z=C_z(\alpha ,\beta, M,q,\delta_v,\delta_P,\delta_e)
\end{equation}
Uglovi $\alpha, \beta$ i $\gamma$ su prikazani na slici \ref{fig:angles} i definisani su sa:
\begin{equation}
    \alpha=arctg(w/u)
\end{equation}
\begin{equation}
    \beta=\arcsin(v/v_m)
\end{equation}
\begin{figure}[h!]
    \centering
    \includegraphics[scale=0.7]{angles.JPG}
    \caption{Ugaone veze}
    \label{fig:angles}
\end{figure}
Razvojem u Taylorov red i odbacivanjem viših članova dobija se aproksimacija 
aerodinamičkih koeficijenata:
\begin{equation}
    C_x=C_{x_0}+C_{x_\alpha}|\alpha|+C_{x_\alpha^2}\alpha^2+C_{x_\beta}|\beta|+
    C_{x_\beta^2}\beta^2+C_{x_\alpha \beta}|\alpha||\beta|
\end{equation}
\begin{equation}
    C_z=C_{z_0}+C_{z_\alpha}|\alpha|+C_{z_\alpha^2}\alpha^2+C_{z_\beta}|\beta|+
    C_{z_\beta^2}\beta^2+C_{z_\alpha \beta}|\alpha||\beta|
\end{equation}
\begin{equation}
    C_y=C_{y_0}+C_{y_\alpha}|\alpha|+C_{y_\alpha^2}\alpha^2+C_{y_\beta}|\beta|+
    C_{y_\beta^2}\beta^2+C_{y_\alpha \beta}|\alpha||\beta|
\end{equation}
U datom slučaju aerodinamički koeficijenti imaju jednostavnijij oblik
\begin{equation}
    C_x=C_{x_0} + C_{x_1}\alpha
\end{equation}
\begin{equation}
    C_z=C_{z_0} + C_{z_1}\alpha
\end{equation}
,gdje je $C_{x_0}=\frac{\partial c_x}{\partial \alpha}|_{\alpha=0}$, $C_{x_1}=\frac{\partial c_x}{\partial \alpha}$ 
i slično tako za ostale izvode. Sada se vraćanjem u \ref{eq:a1},\ref{eq:a2} i \ref{eq:a3} mogu odrediti 
aerodinamičke sile koje djeluju na projektil. 
\section{Aerodinamički momenti}
Momenti se mogu podjeliti na momente koji su posljedica aerodinamičkog tereta i 
pogonske sile koja ne djeulje kroz centar gravitacije. Moment koji je posljedica 
rezultantne sile koja ne djeluje na centar kooridnatnog sistema tijela se može 
podjeliti na tri komponente, i to:
\begin{itemize}
    \item \textbf{Moment valjanja} je moment oko lateralne ose($Y_b$) projektila i generisan 
    je od uzgonom i otporom koje djeluju na tijelo. Pozitivan moment je u smjero gore 
    od nosa letjelice 
    
    \item \textbf{Moment propinjanja} je moment oko longitudinalne ose($X_b$) projektila.
    Posljedica je uzogna koji je uzrokovan nekom vrstom elerona. Pozitivan moment propinjanja uzrokuje kretanje nadole 
    desnog krila.
    
    \item \textbf{Moment zakretanja} je moment oko vetikalne ose projektila($Z_b$). Pozitivan moment zakretanja 
    ima za posljedicu da se nos aviona zakrene u desno. 
\end{itemize}
Kvantitativno, momenti su dati sa:
\begin{equation}
    \text{Moment valjanja} \quad L=C_lqSb
    \label{eq:a1}
 \end{equation}
 \begin{equation}
     \text{Moment propinjanja} \quad M=C_mqSc
     \label{eq:a2}
 \end{equation}
 \begin{equation}
     \text{Moment zakretanja} \quad N=C_nqSb
     \label{eq:a3}
 \end{equation}
 ,gdje je $b$ raspon krila, $c$ je razmak između početne i krajnje ivice krila mjerene 
 u smjeru paralelnom toku vazduha, $S$ je površina platforme krila. 

\chapter{Dinamički model}
Potpun nelinearni dinamički model sastoji iz 12 diferencijalnih jednačina koje su predstavljene ranije. 
Iznimno je teško dobiti analitičko rješenje ovih diferencijalnih jednačina pa se obično pribjegava numeričkoj
simulaciji modela. Zadatak autopilota je da osigura brz prelaz stanja i stabilan odziv u okolini nominalne trajektorije. 
Pokazaće se da se za nominalnu trajektoriju čitav model može raspregnuti što ima za posljedicu 
potpuno razdvajanje modela na dva podsistema. Ova praksa je korištena kod starih letjelica zbog 
uštede računarske moći, ali to danas više nije problem zbog razvoja digitalnih račuara, međutim rasprezanje 
dinamičkog modela je i danas korisno u svrhu sinteze regulatora. Rasprezanje dinamičkog modela 
uvodi netačnosti u model pošto je za rasprezanje potrebno zanemarivanje određenih veličina pa se 
preporučuje ispitivanje regulatora na nelinearnom modelu. U nastavku su sumarno prikazane ranije izvedene 
relacije koje opisuju model projektila krstaste konfiguracije pri čemu treba primjetiti da su ove jednačine 
sada prikazane u koordinatnom sistemu brzine. Korišten je indeks $v$(velocity) da se označi vektor u sistemu brzine
i indeks $b$(body) da se označi vekor u sistemu tijela. Da bi se transformisao vektor iz sistema tijela u sistem brzine 
treba se koristiti inverz matrice transformacije date sa \ref{eq:VtoB}, koji iznosi:
\begin{equation}
    T_b^v = \begin{bmatrix}
            \cos\beta & \sin\beta & 0\\
            -\sin\beta & \cos\beta & 0\\
            0 & 0& 1\\
        \end{bmatrix}
        \begin{bmatrix}
            \cos\alpha & 0 & -\sin\alpha \\
        0& 1& 0\\
        \sin\alpha & 0 & \cos\alpha
        \end{bmatrix}
\end{equation}
Nakon množenja matrica dobija se:
\begin{equation}
    T_b^v\begin{bmatrix}
        \cos\alpha\cos\beta & \sin\beta & -\sin\alpha\cos\beta \\
        -\cos\alpha\sin\beta & \cos\beta & \sin\alpha\sin\beta \\
        \sin\alpha & 0 & \cos\alpha
    \end{bmatrix}
\end{equation}
Sada se konačno može napisati svih 12 diferencijalnih jednačina modela u koordinatnom sistemu brzine.
\begin{align}
    &\frac{dV}{dt} = \frac{F_{xv}}{m}\\
    &\frac{d\Psi}{dt} = \frac{F_{yv}}{mv\cos\Theta}\\
    &\frac{d\Theta}{dt} = -\frac{F_{zv}}{mv}\\
    &\frac{dP}{dt}=L/I_x\\
    &\frac{dQ}{dt}=[M+(I_z-I_x)RP]/I_y\\
    &\frac{dR}{dt}=[N+(I_x-I_y)PQ]/I_z\\
    &\frac{d\psi}{dt}=(R\cos\phi+Q\sin\phi)/\cos\theta\\
    &\frac{d\theta}{dt}=Q\cos\phi-R\sin\phi\\
    &\frac{d\phi}{dt}=P+(R\cos\phi+Q\sin\phi)\tan\theta\\
    &\frac{dx_z}{dt}=V\cos\Theta\cos\Psi\\
    &\frac{dy_z}{dt}=V\cos\Theta\sin\Psi\\
    &\frac{dz_z}{dt}=V\sin\Theta
\end{align}
Ovaj nelinearni model ima tri ulaza(otkloni kontrolnih površina) i svaka od varijabli stana može izlaz pa se kod lineariziranog 
modela može predstaviti 36 prenosnih funkcija, međutim zbog prirode posmatrane konfiguracije 
neke od ovih prenosnih funkcija će identički biti jednake nuli. Jedan primjer ovakve prensone funkcije 
jeste veza između otklona upravljačke površine za stabilizaciju ugla valjanja i brzine projektila. 
\section{Rasprezanje dinamičkog modela}
Sada će se u svrhu lakše analize i sinteze regulatora izvršiti rasprezanje dinamičkog modela. Ideja je 
da se uvedu neke pretpostavke koje će omogućiti da se predstavljene jednačine razdvoje na grupe 
nezavisnih jednačina.Treba da je ispunjeno:
\begin{itemize}
    \item Projektil se kreće u vertikalnoj ravni referentnog koordinatnog sistema.
    \item Osa $x_z$ leži u ravni kretnja.
\end{itemize}
Prva pretpostavka iziskuje $\beta , \phi , P, R\approx 0$. Činjenica da je $P,R \approx 0$ znači da se tijelo rotira samo oko $Y_b$ ose
,dalje, pretpostavka da je $\beta \approx 0$ znači da je usmjerenje letjelica isto kao i vektor brzine i konačno činjenica 
da je $\phi \approx 0$ znači da nema valjanja.  
Druga pretpostavka iziskuje $\Psi, \psi, y_z\approx 0$. Ovo znači da nema skretanja, da projektil može mjenjati samo visinu i udaljenost po $X_z$ osi.
Dakle ove dvije pretpostavke ograničavaju kretanje letjelice na vertikalnu ravan sa dopuštenjem propinjanja i kretanjem naprijed. 
Sada preostale jednačine koje su okarakterisane varijablama stanja:
\[V,\Theta,\theta,\alpha,Q,x_z, z_z\]
Definišu \textit{longitudinalno kretanje(kretanje u vertikalnoj ravni)}. Jednačine okarakterisan varijablama 
koje su u ovom slučaju zanemarene definišu \textit{lateralno kreetanje(bočno)} koje se sastoji 
od kretanja u horizntalnoj ravni, skretanja i valjanja ali bez propinjanja. Sada se ova dva podsistema mogu odvojeno posmatrati.
Sada ostaje samo da se sile koje su objašnjenje u prethodnom poglavlju pretvorimo iz sistema tijela 
i sistema Zemlje u sistem brzine i da ih se uvrsti u jednačine koje opisuju model u sistemu brzine. 
Jednačine koje predstavljaju longitdunalno kretanje su:
\begin{align}
    &m\frac{dv}{dt}= P\cos\alpha - R_{otp} - G\sin\Theta\\
    &mV\frac{d\Theta}{dt} = -Psin\alpha - R_{uzg} + G\cos\Theta\\
    &I_x\frac{dQ}{dt} \approx M\\
    &\frac{d\theta}{dt}=Q\\
    &\frac{dx_z}{dt}=V\cos\Theta\\
    &\frac{dz_z}{dt}=-V\sin\Theta
\end{align}
Također iz matrice transformacije $Tz^v$(treba invertovati $Tv^z$) se dobija za uvedene pretpostavke:
\begin{equation}
    \Theta = \theta - \alpha
\end{equation}
Sada se mogu napisati i jednačine za lateralno kretanje:
\begin{align}
    &mVcos\Theta \frac{d\Psi}{dt}=-P\cos\alpha \sin\beta \cos\phi - P\sin\alpha\sin\phi 
    - R_{side}\cos\phi - R_{uzg}\sin\phi\\
    &I_x\frac{dP}{dt}=L\\
    &I_z\frac{dR}{dt}=N+(I_x-I_y)\\
    &\frac{d\psi}{dt}=(R\cos\phi+q\sin\phi)/\cos\theta\\
    &\frac{d\theta}{dt}=P+(R\cos\phi - q\sin\phi)/\tan\theta\\
    &\frac{dy_z}{dt}=V\cos\Theta\sin\Psi
\end{align}
Pri čemu se iz $Tz^v$ pokazuje:
\begin{equation}
    \Psi \approx \psi-\beta
\end{equation}

\chapter{Uvod u proporcionalnu navigaciju}

\section{Opis planarnog susreta}

\begin{figure}[h!]
    \centering
    \includegraphics{PNfigure.JPG}
\end{figure}

Udaljenost između mete i projektila u svakom trenutku je data sa:
\begin{equation}
    r(t)=r_T(t)-r_M(t)
\end{equation}
Brzina približavanja projektila meti je data sa: 
\begin{equation}
    v_c=-\dot{r}(t)
\end{equation}

Ugaono ubrzanje mete je dato sa:
\begin{equation}
    \dot{\beta}=\frac{n_T}{v_T}
\end{equation}
Kompnente vektora brzine mete u koordinatnom sistemu vezanom za zemlju su date sa:
\begin{equation}
    v_{T1}=-v_T\cos{\beta}
\end{equation}
\begin{equation}
    v_{T2}=v_T\sin{\beta}
\end{equation}
Slično tome, brzina i ubrzanje projektila su date sa:
\begin{equation}
    \dot{v}_{M1}=a_{M1}
\end{equation}
\begin{equation}
    \dot{v}_{M2}=a_{M2}
\end{equation}
\begin{equation}
    \dot{R}_{M1}=v_{M1}
\end{equation}
\begin{equation}
    \dot{R}_{M2}=v_{M2}
\end{equation}
Ugao \textit{Line of sight} se može izračunati kao:
\begin{equation}
    \lambda = \arctan{\frac{R_{TM2}}{R_{TM1}}}
\end{equation}
Pa je: 
\begin{equation}
    \dot{\lambda}=\frac{R_{TM1}v_{TM2}-R_{TM2}v_{TM1}}{r^2}
\end{equation}
Ugao između vektora pozicije i vektora brzine je dat sa:
\begin{equation}
    L=\arcsin{\frac{v_T\sin{(\beta+\lambda)}}{v_M}}
\end{equation}
Također treba uzeti u obzir da je:
\begin{equation}
    v_{cl}=-\dot{r}=v_M\cos\delta - v_T\cos\theta
\end{equation}
Te da će doći do sudara samo u slučaju da vrijedi: 
\begin{equation}
    v_M\cos\delta > v_T\cos\theta
\end{equation}
Upravljački zakon proporcionalne navigacije je dat sa:
\begin{equation}
    n_C=N'v_c\dot{\lambda}
\end{equation}

%-----------------------------
\section{Izvođenje upravljačkog zakona}
\begin{equation}
    \sin{\lambda}=\frac{y}{r}
\end{equation}
Za male uglove može se koristiti aproksimacija:
\begin{equation}
    \lambda \approx \frac{y}{r}
\end{equation}
, pa je:
\begin{equation}
    \dot{\lambda}(t)=\frac{\dot{y}(t)r(t)-y(t)\dot{r}(t)}{r^2}
\end{equation}
\begin{equation}
    \ddot{\lambda}(t)=\frac{\ddot{y}(t)-2\dot{\lambda}(t)\dot{r}(t)-\lambda(t)\ddot{r}(t)}{r(t)}
\end{equation}
Uvedimo vremenski varijantne koeficijente:
\begin{equation}
    a_1(t)=\frac{\ddot{r}(t)}{r(t)}
\end{equation}
\begin{equation}
    a_2(t)=2\frac{\dot{r}(t)}{r(t)}
\end{equation}
\begin{equation}
    b(t)=\frac{1}{r(t)}
\end{equation}
Pa se dobija diferencijalna jednačina drugog reda sa varijabilnim koeficijenitma:
\begin{equation}
    \ddot{\lambda}(t)=-a_1(t)\lambda-a_2(t)\dot{\lambda}+b(t)\ddot{y}(t)
\end{equation}
Uzimajući u obzir dobija se:
\begin{equation}
    \ddot{y}(t)=-a_M(t)+a_T(t)
\end{equation}
\begin{equation}
    \ddot{\lambda}(t)=-a_1(t)\lambda-a_2(t)\dot{\lambda}-b(t)a_M(t)+b(t)a_T(t)
\end{equation}
Neka je $x_1(t)=\lambda$ i $x_2(t)=\dot{\lambda}$. Tada je susret projektila i mete opisan sljdećim diferencijalnim jednačinama prvog reda.
\begin{equation}
    \dot{x}_1=x_2
    \label{eq:1}
\end{equation}
\begin{equation}
    \dot{x}_2=-a_1(t)x_1-a_2(t)x_2-b(t)u+b(t)f
    \label{eq:2}
\end{equation}
,gdje je uzeto $u=a_M(t)$ i vanjska smetnja $f=a_T(t)$.
Prvo posmatrajmo slučaj kada meta ne ubrzava, tj. kada je $f=0$. Sada se problem proporcionalne navigacije može predstaviti kao:
%\subsection{SubSection Title}
\begin{tcolorbox}
    Pronaći upravljački signal $u$ tako da je sistem opisan jednačinama \ref{eq:1} i \ref{eq:2} asimptotski stabilan u odnosu na $x_2$
\end{tcolorbox}

Shodno tome, uzmimo Lyapunovu funkciju $Q$:
\begin{equation}
    Q=\frac{1}{2}cx_2^2
\end{equation}
Izvod po vremenu duž bilo koje trajektorije je:
\begin{equation}
    \dot{Q}=cx_2(-a_1(t)x_1-a_2(t)x_2-b(t)u(t))
\end{equation}
Sada se vidi da upravljački signal 
\begin{equation}
    u=kx_2=k\dot{\lambda}
    \label{eq:3}
\end{equation}
Stabilizuje sistem dat sa \ref{eq:1} i \ref{eq:2} ako $k$ zadovoljava:
\begin{equation}
    kb(t)+a_2(t)>0
\end{equation}
,odnosno \begin{equation}
    k>-2\dot{r}(t)=2v_{cl}
\end{equation}
Prema tome, uvodeći \textit{efektivni navigacijski odnos} $N$, izraz \ref{eq:3} postaje:
\begin{equation}
    u=Nv_{cl}\dot{\lambda}(t) \quad ,N>2
\end{equation}
čime je potpuno određen zakon vođenja proporcionalne navigacije.
Za trodimenzionalni slučaj se bira kandidat funkcija:
\begin{equation}
    Q=\frac{1}{2}\sum_{s=1}^3d_s\dot{\lambda}_s^2
\end{equation}
, gdje su $d_s$ pozitivni koeficijenti. Analogno se dobija upravljački zakon:
\begin{equation}
    u_s=Nv_{cl}\dot{\lambda}_s \quad ,N>2\ (s=1,2,3)
\end{equation}

\section{Izmjenjena proporcionalna navigacija}
Za mete koje manevrišu i imaju neko normalno ubrzanje, za planarni sustre, izvod Lyapunove kandidat 
funkcije je:
\begin{equation}
    \dot{Q}=cx_2(-a_1(t)x_1-a_2(t)x_2-b(t)u(t)+b(t)f)
\end{equation}
Odakle se zaključuje da je upravljaki signal koji stabilizuje sistem:
\begin{equation}
    u=Nv_{cl}\dot{\lambda}(t)+a_T(t) \quad ,N>2
\end{equation}
\section{Optimalnost zakona proporcionalne navigacije}
Ako je promjena LOS ugla različita od nule, tada se primjenjuje normalno ubrzanje kako bi 
se promjena svela na nulu. U prethodnoj sekciji se proporcionalna navigacija predstavila kao 
problem upravljanja gdje je normalno ubrzanje bilo upravljački signal, a brzina promjene LOS ugla bila varijabla stanja.
Proporcionalna navigacija se može posmatrati kao problem optimalnog upravljanja. Treba pronaći indeks performansi koji 
proporcionalna navigacija minimizira. Ovo predstavlja inverzni problem problem optimalnog upravljanja. Pretpostavimo da 
se projektil približava meti konstantnom brzinom. Ignorišuči dinamiku projektila, vrijedi:
\begin{equation}
    \ddot{y}=-a_M,\ y=r\lambda,\ r(\tau)=v_{cl}\tau
\end{equation}
Također pretpostavlja se da nema kašnjenja u dinamici projektila, tj. da je $a_M = a_{M_c}$.
Definišimo sada ineks performansi:
\begin{equation}
    J=\frac{1}{2}Cy^2(t_f)+\frac{1}{2}\int_0^{t_f}{a_M^2dt}
\end{equation}
Prvi član predstavlja promašaj(miss distance), a drugi predstavlja energiju energiju utrošenu u toku leta. Ideja je pronaći upravljanje
$a_M$ koje minimizira kriterij performanse $J$. Koriteći Bellman-Lyapunov pristup dobija se da je 
optimalno upravljanje dato sa:
\begin{equation}
    a_M(t)=\frac{3\tau}{3/C+\tau ^3}(y(t)+\dot{y}(t)\tau)
\end{equation}
Nulti promašaj se dobija za $C\rightarrow \infty$, pa je optimalno upravljanje dato sa:
\begin{equation}
    a_M(t)=\frac{3}{\tau ^2}(y(t)+\dot{y}(t)\tau)
\end{equation}
Uzimajući u obzir da je:
\begin{equation}
    \dot{\lambda} = \frac{\dot{y}(t)r(t)-y(t)\dot{t}(t)}{r^2}=\frac{\dot{y}(t)\tau + y(t)}{r}
\end{equation}
jer je, $r=v_{cl}\tau$, dobija se:
\begin{equation}
    a_M(t)=3v_{cl}\dot{\lambda}
\end{equation}
Ovo znači da pod uvedenim pretpostavkama, proporcionalna navigacija minimizira kriterij performanse
$J$ i izbor efektivnog navigacijskog odnosa $N=3$ garantuje da nulti promašaj. 
%-----------------------
\section{Linearizacija}
\begin{figure}[h!]
    \centering
    \includegraphics{linearPN.JPG}
    \caption{Linearizacija jednačina proporcionalne navigacije}
    \label{fig:linear}
\end{figure}

Linearizacija se može lahko izvršiti ako se definišu nove veličine koje su prikazane na slici \ref{fig:linear}.
Relativno ubrzanje se može odrediti sa slike i iznosi:
\begin{equation}
    \ddot{y}=n_T\cos\beta-n_c\cos\lambda
\end{equation}
Ako su uglovi leta mali, tada vrijedi:
\begin{equation}
    \ddot{y}=n_T-n_c
\end{equation}
Slično tako vrijedi:
\begin{equation}
    \lambda = \frac{y}{r}
\end{equation}
Za čeoni slučaj vrijedi:
\begin{equation}
    v_{cl}=v_M+v_t
\end{equation}
Za potjeru vrijedi:
\begin{equation}
    v_{cl}=v_M-v_t
\end{equation}
Sada se može linearizirati i jednačina za udaljenost:
\begin{equation}
    r(t)=v_{cl}(t_F-t)
\end{equation}
gdje je $t_F$ ukupno vrijeme leta.\\
Definišimo i veličinu \textit{time to go} $t_{go}$:
\begin{equation}
    t_{go}=t_F-t
\end{equation}
Linearizirani promašaj se definisše kao udaljenost mete i projektila na kraju leta, ili:
\begin{equation}
    Miss=y(t_f)
\end{equation}

\section{Zero effort miss}
Ranije je pokazano da vrijedi:
\begin{equation}
    \dot{\lambda}(t)=\frac{\dot{y}(t)r(t)+y(t)v_{cl}}{r^2}
\end{equation}
Kako vrijedi $r=v_{cl}t_{go}$, tada se dobija:
\begin{equation}
    \dot{\lambda}(t)=\frac{\dot{y}(t)t_{go}+y(t)}{v_{cl}t_{go}^2}
\end{equation}
Definišimo sada veličinu \textit{Zero effort miss}, koja predstavlja buduće relativno rastojanje projektila i mete:
\begin{equation}
    ZEM=\dot{y}(t)t_{go}+y(t)
\end{equation}
pa se dobija:
\begin{equation}
    \dot{\lambda}(t)=\frac{ZEM}{v_{cl}t_{go}^2}
\end{equation}
Ako se pretpstavi da će se pod uticajem ubrzanja $a_c$ postići sudar, $ZEM$ se može smatrati 
budućom tačkom susreta, pa se zakon vođenja proprcionalne navigacije može iskazati kao:
\begin{equation}
    a_c(t)=N\frac{ZEM}{t_{go}^2}
\end{equation}
Sada se vidi da je normalno ubrzanje projektila direktno proprorcionalnu $ZEM$-u i inverzno proporcionalno
kvadratu preostalom vremenu leta, što znači da se generiše veće ubrzanje što je susret bliži.
Pošto se $ZEM$ posmatra kao buduća tačka susreta, koja se računa na osnovu znanja ili pretpostavki 
budučeg kretanja mete, PN vođenje se smatra prediktivnim. $ZEM$ je koristan jer se može izračunati 
mnoštvom metoda uključujući i on-line numeričku integraciju nelinearnih diferencijalnih jednačina projektila 
i mete. 

\begin{figure}[htp]
    \centering
    \includegraphics{homingLoop.JPG}
    \label{fig:homing}
\end{figure}

\chapter{Sinteza autopilota}
Autopilot je sistem sa zatvorenom povratnom spregom unutar sistema za vođenje objekta 
u prostoru koji osigurava da projektil dostigne ubrzanje koje mu sistem vođenja zapovjeda. 
Funkcija autopilota je da stabilizuje i vodi projektil tako što zadaje upravljačke signale kontrolnim 
površinama koji tjeraju projkeil da se rotira ondnosno da translira.  
Pošto tranzijentni odziv projektila varira sa promjenom uslova leta, tako i parametri 
autopilota treba da se mjenjaju sa uslovima leta pa prema tome dobro 
dizajniran autopilot osigurava skoro linearan odziv. Najčešće se za projektovanje 
autopilota koristi linerizirani model drugog reda koji je ranije izveden. 
\section{Upravljanje i stabilizacija ugla propinjanja}
\section{Upravljanje normalnim ubrzanjem}
\section{Three loop autopilot}
\chapter{Zaključak}
Za razumijevanje problematike vođenja projektila potrebo je više koordinatnih sistema. 
Ostvaren je model projektila sa šest stepeni slobode u koordinatnom sistemu tijela. 
Model projektila se sastoji iz tri nezavisna kanala: kanal visine, kanal pravca i kanal valjanja.
Pokazano je da moguće ostvariti vođenje projektila ka meti regulacijom kanala visine i pravca i 
stabilizacijom kanala valjanja. Za vođenje projektil koristi se zakon proporcionalne navigacije 
koji generiše ubrzanja u referentnom sistemu koji garantuju da će se projektil susresti sa metom.
Prikazan je potpun sistem vođenja zajedno sa autopilotom i pokazano je da 
se ostvaruje pogodak čak i ako postoji razumna početna greška nišanjenja. 


\nocite{*}
\printbibliography

\end{document}