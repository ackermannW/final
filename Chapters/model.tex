Potpun nelinearni dinamički model sastoji iz 12 diferencijalnih jednačina koje su predstavljene ranije. 
Iznimno je teško dobiti analitičko rješenje ovih diferencijalnih jednačina pa se obično pribjegava numeričkoj
simulaciji modela. Zadatak autopilota je da osigura brz prelaz stanja i stabilan odziv u okolini nominalne trajektorije. 
Pokazaće se da se za nominalnu trajektoriju čitav model može raspregnuti što ima za posljedicu 
potpuno razdvajanje modela na dva podsistema. Ova praksa je korištena kod starih letjelica zbog 
uštede računarske moći, ali to danas više nije problem zbog razvoja digitalnih račuara, međutim rasprezanje 
dinamičkog modela je i danas korisno u svrhu sinteze regulatora. Rasprezanje dinamičkog modela 
uvodi netačnosti u model pošto je za rasprezanje potrebno zanemarivanje određenih veličina pa se 
preporučuje ispitivanje regulatora na nelinearnom modelu. U nastavku su sumarno prikazane ranije izvedene 
relacije koje opisuju model projektila krstaste konfiguracije pri čemu treba primjetiti da su ove jednačine 
sada prikazane u koordinatnom sistemu brzine. Korišten je indeks $v$(velocity) da se označi vektor u sistemu brzine
i indeks $b$(body) da se označi vekor u sistemu tijela. Da bi se transformisao vektor iz sistema tijela u sistem brzine 
treba se koristiti inverz matrice transformacije date sa \ref{eq:VtoB}, koji iznosi:
\begin{equation}
    T_b^v = \begin{bmatrix}
            \cos\beta & \sin\beta & 0\\
            -\sin\beta & \cos\beta & 0\\
            0 & 0& 1\\
        \end{bmatrix}
        \begin{bmatrix}
            \cos\alpha & 0 & -\sin\alpha \\
        0& 1& 0\\
        \sin\alpha & 0 & \cos\alpha
        \end{bmatrix}
\end{equation}
Nakon množenja matrica dobija se:
\begin{equation}
    T_b^v\begin{bmatrix}
        \cos\alpha\cos\beta & \sin\beta & -\sin\alpha\cos\beta \\
        -\cos\alpha\sin\beta & \cos\beta & \sin\alpha\sin\beta \\
        \sin\alpha & 0 & \cos\alpha
    \end{bmatrix}
\end{equation}
Sada se konačno može napisati svih 12 diferencijalnih jednačina modela u koordinatnom sistemu brzine.
\begin{align}
    &\frac{dV}{dt} = \frac{F_{xv}}{m}\\
    &\frac{d\Psi}{dt} = \frac{F_{yv}}{mv\cos\Theta}\\
    &\frac{d\Theta}{dt} = -\frac{F_{zv}}{mv}\\
    &\frac{dP}{dt}=L/I_x\\
    &\frac{dQ}{dt}=[M+(I_z-I_x)RP]/I_y\\
    &\frac{dR}{dt}=[N+(I_x-I_y)PQ]/I_z\\
    &\frac{d\psi}{dt}=(R\cos\phi+Q\sin\phi)/\cos\theta\\
    &\frac{d\theta}{dt}=Q\cos\phi-R\sin\phi\\
    &\frac{d\phi}{dt}=P+(R\cos\phi+Q\sin\phi)\tan\theta\\
    &\frac{dx_z}{dt}=V\cos\Theta\cos\Psi\\
    &\frac{dy_z}{dt}=V\cos\Theta\sin\Psi\\
    &\frac{dz_z}{dt}=V\sin\Theta
\end{align}
Ovaj nelinearni model ima tri ulaza(otkloni kontrolnih površina) i svaka od varijabli stana može izlaz pa se kod lineariziranog 
modela može predstaviti 36 prenosnih funkcija, međutim zbog prirode posmatrane konfiguracije 
neke od ovih prenosnih funkcija će identički biti jednake nuli. Jedan primjer ovakve prensone funkcije 
jeste veza između otklona upravljačke površine za stabilizaciju ugla valjanja i brzine projektila. 
\section{Rasprezanje dinamičkog modela}
Sada će se u svrhu lakše analize i sinteze regulatora izvršiti rasprezanje dinamičkog modela. Ideja je 
da se uvedu neke pretpostavke koje će omogućiti da se predstavljene jednačine razdvoje na grupe 
nezavisnih jednačina.Treba da je ispunjeno:
\begin{itemize}
    \item Projektil se kreće u vertikalnoj ravni referentnog koordinatnog sistema.
    \item Osa $x_z$ leži u ravni kretnja.
\end{itemize}
Prva pretpostavka iziskuje $\beta , \phi , P, R\approx 0$. Činjenica da je $P,R \approx 0$ znači da se tijelo rotira samo oko $Y_b$ ose
,dalje, pretpostavka da je $\beta \approx 0$ znači da je usmjerenje letjelica isto kao i vektor brzine i konačno činjenica 
da je $\phi \approx 0$ znači da nema valjanja.  
Druga pretpostavka iziskuje $\Psi, \psi, y_z\approx 0$. Ovo znači da nema skretanja, da projektil može mjenjati samo visinu i udaljenost po $X_z$ osi.
Dakle ove dvije pretpostavke ograničavaju kretanje letjelice na vertikalnu ravan sa dopuštenjem propinjanja i kretanjem naprijed. 
Sada preostale jednačine koje su okarakterisane varijablama stanja:
\[V,\Theta,\theta,\alpha,Q,x_z, z_z\]
Definišu \textit{longitudinalno kretanje(kretanje u vertikalnoj ravni)}. Jednačine okarakterisan varijablama 
koje su u ovom slučaju zanemarene definišu \textit{lateralno kreetanje(bočno)} koje se sastoji 
od kretanja u horizntalnoj ravni, skretanja i valjanja ali bez propinjanja. Sada se ova dva podsistema mogu odvojeno posmatrati.
Sada ostaje samo da se sile koje su objašnjenje u prethodnom poglavlju pretvorimo iz sistema tijela 
i sistema Zemlje u sistem brzine i da ih se uvrsti u jednačine koje opisuju model u sistemu brzine. 
Jednačine koje predstavljaju longitdunalno kretanje su:
\begin{align}
    &m\frac{dv}{dt}= P\cos\alpha - R_{otp} - G\sin\Theta\\
    &mV\frac{d\Theta}{dt} = -Psin\alpha - R_{uzg} + G\cos\Theta\\
    &I_x\frac{dQ}{dt} \approx M\\
    &\frac{d\theta}{dt}=Q\\
    &\frac{dx_z}{dt}=V\cos\Theta\\
    &\frac{dz_z}{dt}=-V\sin\Theta
\end{align}
Također iz matrice transformacije $Tz^v$(treba invertovati $Tv^z$) se dobija za uvedene pretpostavke:
\begin{equation}
    \Theta = \theta - \alpha
\end{equation}
Sada se mogu napisati i jednačine za lateralno kretanje:
\begin{align}
    &mVcos\Theta \frac{d\Psi}{dt}=-P\cos\alpha \sin\beta \cos\phi - P\sin\alpha\sin\phi 
    - R_{side}\cos\phi - R_{uzg}\sin\phi\\
    &I_x\frac{dP}{dt}=L\\
    &I_z\frac{dR}{dt}=N+(I_x-I_y)\\
    &\frac{d\psi}{dt}=(R\cos\phi+q\sin\phi)/\cos\theta\\
    &\frac{d\theta}{dt}=P+(R\cos\phi - q\sin\phi)/\tan\theta\\
    &\frac{dy_z}{dt}=V\cos\Theta\sin\Psi
\end{align}
Pri čemu se iz $Tz^v$ pokazuje:
\begin{equation}
    \Psi \approx \psi-\beta
\end{equation}