Potpun nelinearni dinamički model sastoji iz 12 diferencijalnih jednačina koje su predstavljene ranije. 
Iznimno je teško dobiti analitičko rješenje ovih diferencijalnih jednačina pa se obično pribjegava numeričkoj
simulaciji modela. Zadatak autopilota je da osigura brz prelaz stanja i stabilan odziv u okolini nominalne trajektorije. 
Pokazaće se da se za nominalnu trajektoriju čitav model može raspregnuti što ima za posljedicu 
potpuno razdvajanje modela na dva podsistema. Ova praksa je korištena kod starih letjelica zbog 
uštede računarske moći, ali to danas više nije problem zbog razvoja digitalnih račuara, međutim rasprezanje 
dinamičkog modela je i danas korisno u svrhu sinteze regulatora. Rasprezanje dinamičkog modela 
uvodi netačnosti u model pošto je za rasprezanje potrebno zanemarivanje određenih veličina pa se 
preporučuje ispitivanje regulatora na nelinearnom modelu. U nastavku su sumarno prikazane ranije izvedene 
relacije koje opisuju model projektila krstaste konfiguracije pri čemu treba primjetiti da su ove jednačine 
sada prikazane u koordinatnom sistemu brzine. Korišten je indeks $v$(velocity) da se označi vektor u sistemu brzine
i indeks $b$(body) da se označi vekor u sistemu tijela. Da bi se transformisao vektor iz sistema tijela u sistem brzine 
treba se koristiti inverz matrice transformacije date sa \ref{eq:VtoB}, koji iznosi:
\begin{equation}
    T_b^v = \begin{bmatrix}
            \cos\beta & \sin\beta & 0\\
            -\sin\beta & \cos\beta & 0\\
            0 & 0& 1\\
        \end{bmatrix}
        \begin{bmatrix}
            \cos\alpha & 0 & -\sin\alpha \\
        0& 1& 0\\
        \sin\alpha & 0 & \cos\alpha
        \end{bmatrix}
\end{equation}
Nakon množenja matrica dobija se:
\begin{equation}
    T_b^v\begin{bmatrix}
        \cos\alpha\cos\beta & \sin\beta & -\sin\alpha\cos\beta \\
        -\cos\alpha\sin\beta & \cos\beta & \sin\alpha\sin\beta \\
        \sin\alpha & 0 & \cos\alpha
    \end{bmatrix}
\end{equation}
Sada se konačno može napisati svih 12 diferencijalnih jednačina modela u koordinatnom sistemu brzine.
\begin{align}
    &\frac{dV}{dt} = \frac{F_{xv}}{m}\\
    &\frac{d\Psi}{dt} = \frac{F_{yv}}{mv\cos\Theta}\\
    &\frac{d\Theta}{dt} = -\frac{F_{zv}}{mv}\\
    &\frac{dP}{dt}=L/I_x\\
    &\frac{dQ}{dt}=[M+(I_z-I_x)RP]/I_y\\
    &\frac{dR}{dt}=[N+(I_x-I_y)PQ]/I_z\\
    &\frac{d\psi}{dt}=(R\cos\phi+Q\sin\phi)/\cos\theta\\
    &\frac{d\theta}{dt}=Q\cos\phi-R\sin\phi\\
    &\frac{d\phi}{dt}=P+(R\cos\phi+Q\sin\phi)\tan\theta\\
    &\frac{dx_z}{dt}=V\cos\Theta\cos\Psi\\
    &\frac{dy_z}{dt}=V\cos\Theta\sin\Psi\\
    &\frac{dz_z}{dt}=V\sin\Theta
\end{align}
Ovaj nelinearni model ima tri ulaza(otkloni kontrolnih površina) i svaka od varijabli stana može izlaz pa se kod lineariziranog 
modela može predstaviti 36 prenosnih funkcija, međutim zbog prirode posmatrane konfiguracije 
neke od ovih prenosnih funkcija će identički biti jednake nuli. Jedan primjer ovakve prensone funkcije 
jeste veza između otklona upravljačke površine za stabilizaciju ugla valjanja i brzine projektila. 
\section{Rasprezanje dinamičkog modela}
Sada će se u svrhu lakše analize i sinteze regulatora izvršiti rasprezanje dinamičkog modela. Ideja je 
da se uvedu neke pretpostavke koje će omogućiti da se predstavljene jednačine razdvoje na grupe 
nezavisnih jednačina.Treba da je ispunjeno:
\begin{itemize}
    \item Projektil se kreće u vertikalnoj ravni referentnog koordinatnog sistema.
    \item Osa $x_z$ leži u ravni kretnja.
\end{itemize}
Prva pretpostavka iziskuje $\beta , \phi , P, R\approx 0$. Činjenica da je $P,R \approx 0$ znači da se tijelo rotira samo oko $Y_b$ ose
,dalje, pretpostavka da je $\beta \approx 0$ znači da je usmjerenje letjelica isto kao i vektor brzine i konačno činjenica 
da je $\phi \approx 0$ znači da nema valjanja.  
Druga pretpostavka iziskuje $\Psi, \psi, y_z\approx 0$. Ovo znači da nema skretanja, da projektil može mjenjati samo visinu i udaljenost po $X_z$ osi.
Dakle ove dvije pretpostavke ograničavaju kretanje letjelice na vertikalnu ravan sa dopuštenjem propinjanja i kretanjem naprijed. 
Sada preostale jednačine koje su okarakterisane varijablama stanja:
\[V,\Theta,\theta,\alpha,Q,x_z, z_z\]
Definišu \textit{longitudinalno kretanje(kretanje u vertikalnoj ravni)}. Jednačine okarakterisan varijablama 
koje su u ovom slučaju zanemarene definišu \textit{lateralno kreetanje(bočno)} koje se sastoji 
od kretanja u horizntalnoj ravni, skretanja i valjanja ali bez propinjanja. Sada se ova dva podsistema mogu odvojeno posmatrati.
Sada ostaje samo da se sile koje su objašnjenje u prethodnom poglavlju pretvorimo iz sistema tijela 
i sistema Zemlje u sistem brzine i da ih se uvrsti u jednačine koje opisuju model u sistemu brzine. 
Jednačine koje predstavljaju longitdunalno kretanje su:
\begin{align}
    &m\frac{dv}{dt}= P\cos\alpha - R_{otp} - G\sin\Theta\\
    &mV\frac{d\Theta}{dt} = -Psin\alpha - R_{uzg} + G\cos\Theta\\
    &I_x\frac{dQ}{dt} \approx M\\
    &\frac{d\theta}{dt}=Q\\
    &\frac{dx_z}{dt}=V\cos\Theta\\
    &\frac{dz_z}{dt}=-V\sin\Theta
\end{align}
Također iz matrice transformacije $Tz^v$(treba invertovati $Tv^z$) se dobija za uvedene pretpostavke:
\begin{equation}
    \Theta = \theta - \alpha
\end{equation}
Sada se mogu napisati i jednačine za lateralno kretanje:
\begin{align}
    &mVcos\Theta \frac{d\Psi}{dt}=-P\cos\alpha \sin\beta \cos\phi - P\sin\alpha\sin\phi 
    - R_{side}\cos\phi - R_{uzg}\sin\phi\\
    &I_x\frac{dP}{dt}=L\\
    &I_z\frac{dR}{dt}=N+(I_x-I_y)\\
    &\frac{d\psi}{dt}=(R\cos\phi+q\sin\phi)/\cos\theta\\
    &\frac{d\theta}{dt}=P+(R\cos\phi - q\sin\phi)/\tan\theta\\
    &\frac{dy_z}{dt}=V\cos\Theta\sin\Psi
\end{align}
Pri čemu se iz $Tz^v$ pokazuje:
\begin{equation}
    \Psi \approx \psi-\beta
\end{equation}
Još uvijek se nisu u diferencijalne jednačine uvele linearizirane vrijednosti za aerodinamičke 
sile i momente pa se u jednačinama ne pojavljuju upralvjačke varijable, zbog toga će se u nastavku uraditi
poptuna linearizacija dinamičkog modela. Tada će se dobiti zavisnost varjiabli stanja od ulaza, pa 
je na osnovu toga moguće riješiti ove jednačine da bi se odredile varijable stanja. Iz ovoga slijedi i obrat 
tj. da se mogu odrediti otkloni upralvjačkih površina da bi se postigle željene vrijednosti varijabli stanja 
koje zahtjeva zakon voođenja. Naravno ovakav postupak je u otvorenoj petlji pa se zbog netačnosti modela 
preporučuje upravljanje u zatvorenoj povratnoj sprezi. 
\section{Linearizacija u okolini nominalne trajektorije}
Generalno, kada se priča o linearizaciji sistema, radi se o linearizaciji oko neke radne tačke. Ideja je 
da se diferencijalna jednačina u okolini te radne tačke predstavi linearnim segmentnom, te da 
nakon toga ona ima linearnu zavisnot od ulaznih parametara. Kod kretanja projektila umjesto pojma 
radne tačke se uvodi pojam \textit{nominalne trajektorije}. To je trajektorija po kojoj projektil 
leti kada su sve varijable stanja upravo onakve kako se od njih očekuje da budu i kada nema vanjskih poremećaja 
na projektil osim aerodinamičkog otpora i gravitacije. Sada se kao suprotnost nominalnoj trajektoriji 
uvodi pojam \textit{pormećajnog kretanja} koje se odlikuje odstupanjem varijabli stanja od nominalnih vrijednosti. 
Pri ovome se pretpostavlja da su odstupanja varijabli stanja pri poremećajnom kretanju relativno mala u odnosu na 
njihove nominalne vrijednosti. Svaka nominalna trajektorija određena je nekom vrijednošću vektora stanja $X_{nom}$.
Do ostalih vrijednosti može se doći rješavanjem jednačine:
\begin{equation}
    \dot{\vec{X}}_{nom} = f(\vec{X}_{nom}, \delta_{nom})
\end{equation}
Sada će se izvšiti linearizacija modela longitudinalnog kretanja. Pretpostavlja se da u okolini radne tačke, 
vrijednosti varijabli stanja imaju oblik:
\begin{equation}
    x=x_0+\Delta x 
\end{equation}
Prisjetimo se samo da u okolini nominalne trajektorije upravlja;ki signal može definisati kao:
\begin{equation}
    u(t) = u_0(t)+\Delta u(t)
\end{equation}
Pa je:
\begin{equation}
    \dot{x}_0(t)+\Delta \dot{x}(t)=f(x_0(t)+\Delta x(t),u_0(t)+\Delta u(t))
\end{equation}
Funkcija na desnoj strani se može raziti u Taylorov red i nakon odbacivanja članova višeg reda
se dobija:
\begin{equation}
    \dot{x}_0(t)+\Delta \dot{x}(t)=f(x_0(t),u_0(t))+\frac{\partial f}{\partial x}\Delta x+\frac{\partial f}{\partial u}\Delta u
\end{equation}
Sada se može napisati:
\begin{equation}
    \Delta \dot{x}(t)=\frac{\partial f}{\partial x}\Delta x+\frac{\partial f}{\partial u}\Delta u
\end{equation}
Parcijalni izvodi se uzimaju tako da vrijedi $x=x_0$ i $u=u_0$.\\
Kod modela longitudinalnog kretanja će se izvršiti isti postupak s tim da će se linearizirati svaka 
jednačina posebno. Sada za model longitduinalnog kretanja, ako se pretpostavi da se 
projektil kreće po nominalnoj trajektiri, vrijede jednačine:
\begin{align}
    &V=V_0+\Delta V \\
    & \alpha = \alpha _0+\Delta \alpha\\
    & \Theta=\Theta _0 +\Delta \Theta\\
    & \theta= \theta _0+\Delta \theta\\
    & Q=Q_0+\Delta Q\\
    & z_z=z_{z0}+\Delta z_z\\
    & \delta _V=\delta _{V0}+\Delta \delta _V
\end{align}
Koristeći pretpostavku da je $\cos\alpha_0 \approx 1$ i koristeći gore predstavljenu metodologiju 
linearizacije može se dobiti:
\begin{align}
    &\frac{d\Delta V}{dt}=\frac{P^V-F_o^V}{m}\Delta V - \frac{P\alpha+F_0^\alpha}{m}\Delta\alpha - g\cos\Theta _0\Delta\Theta +\frac{F_u^{\delta )V}}{m}\delta_V+\frac{X_P}{m}\\
    &\frac{d\Delta \Theta}{dt}=\frac{P^V-F_u^V}{m}\Delta V + \frac{P-F_u^\alpha}{mV}\Delta \alpha-\frac{g}{V}\sin\Theta\Delta\Theta-\frac{F_u^{\delta_V}}{\Delta_V}+\frac{Z_P}{mV}\\
    &\frac{d\Delta Q}{dt}=\frac{M^V}{I_y}\Delta V+\frac{M^\alpha}{I_y}\Delta Q+\frac{M^{\dot{\alpha}}}{I_y}\Delta\dot{\alpha}+\frac{M^{\delta_V}}{I_y}\Delta \delta_V+\frac{M^{\dot{\delta}_V}}{I_y}\Delta \dot{\delta}_V+\frac{M_P}{I_y} \\
    &\frac{d\Delta \theta}{dt}=\Delta Q \\
    &\frac{d\Delta x_z}{dt}=\cos\Theta_0\Delta V-V\sin\Theta_0\Delta\Theta\\
    &\frac{d\Delta z_z}{dt}=\sin\Theta_0\Delta V+V\cos\Theta_0\Delta\Theta \\
    &\Delta\alpha=\Delta\theta - \Delta\Theta
\end{align}
U koeficijentima dobijenih diferencijalnih jednačina su exponentima označeni izvodi te veličine. Konkretno, 
$P^V=\frac{\partial P}{\partial V}$, $F_o^\alpha = \frac{\partial F_o}{\partial \alpha} = QSC_o^\alpha$ etc. 
Svi ovi parcijalni izvodi su objašnjeni kada se govorilo o prirodi aerodinamičkih sila i momenata i oni se često 
za projektil daju tabelarno. Članovi $X_P$, $Z_P$ i $M_P$ predstavljaju poremećaje u vidu sila i momenata i oni ovdje 
djelom predstavljaju ulaze u sistem. Sada se u ovim jednačinama po prvi put eksplicitno vide upravljačke varijable.  
Na isti način se mogu naći i lineariziane jednačine za lateralno kretanje.
Ako se nađe Laplasova transformacija gornjih jednačina, rješavanjem dobijenog sistema algebarskih jednačina 
dobija se karakteristični polinom funkcija prenosa(sjetimo se da kod MIMO sistema, sve prenosne funkcije imaju isti karakteristični polinom). 
Radi se o polinomu četvrtog reda kod kojeg je jedan par polova po modulu dosta veći od drugog para polova po modulu. 
Sada je jasno da se kretanje letjelice može razdovjiti na \textit{brzo prigušeno} kretanje koje može biti oscilatorno 
ili aperiodičko i na \textit{fugoidno(sporo prigušeno)}. Dinamiku modela longitudinalnog kretanja određuje 
dominantni par polova koji je manji po modulu pa je kretanje letjelice određeno fugoidnim kretanje. Sada je jasno da se 
polovi koji opsiuju brzoprigušeno kretanje mogu odbaciti pa će karakterisični polinom imati samo dva pola. 
Dakle, sada se posmatraju samo jednačine koje opisuju kratkoperiodično kretanje. Brzoperiodično kretanje je 
određeno jednačinom promjene brzine(prva diferencijalna jednačina) pa se nakon uvođenja ove pretpostavke odbacuje ova 
jednačina  i u ostalim se anulira $\Delta V$. Sada teba primjetiti da se u lineariziranim jednačinama pojavljuje koeficijent 
$-\frac{g}{V}\sin\Theta_0$. Ovaj koeficijent predstavlja uticaj gravitacije na longitudinalno kretanje. Za male elevacione 
uglove, ovaj koeficijent je jako blizak nuli. Čak i kada trajektorija puno odstupa od horizontalne, brzina projektila 
je najmanje 20 puta veća od gravitacionog ubrzanja pa se uticaj gravitacije na longitudinalno kretanje može zanemariti. 
Ova pretpostavka u prenosnim funkcijama uvodi pol u nuli, tj. pod ovom pretpostavkom sistem će se ponašati kao integrator i 
sam će osiguratio nultu grešku stacionarnog stanja. Međutim ako ova pretpostavka nije ispunjenja tada će se pojaviti pol blizak nuli, 
pa će prelazni proces biti dug možda čak i nestabilan. Sada, pod ovim pretpostavkama dobijaju sljedeće prenosne funkcije:
\begin{align}
    &\frac{\Delta \theta(s)}{\Delta \delta_V(s)}=\frac{K(T_1s+1)}{s(T^2s^2+2\xi Ts+1)}\\
    &\frac{\Delta \Theta(s)}{\Delta \delta_V(s)}=\frac{K}{s(T^2s^2+2\xi Ts+1)}\\
    &\frac{\Delta \Theta(s)}{\Delta \delta_V(s)}=\frac{KT_1}{T^2s^2+2\xi Ts+1}\\
    &\frac{\Delta n_z(s)}{\Delta \delta_V(s)}=\frac{V}{g}\frac{K}{T^2s^2+2\xi Ts+1}
\end{align}
,gdje $n_z = \frac{V\dot{\Theta}}{g}$ predstavlja \textit{normalno preopterećenje}, tj. odnos ubrzanja koje je normalno na pravac brzine i gravitacione konstante. 
Evidentno je da je normalno ubrzanje definisano izrazom:
\begin{equation}
    a_z = V\dot{\Theta}
\end{equation}
I predstavlja jako bitnu veličinu jer mnogi zakoni vođenja generišu komandne signale u vidu normalnog ubrzanja projektil, pa će se i 
posebna pažnja posvetiti upravljanju normalnog ubrzanja. Prenosna funkcija koja određuje normalno ubrzanje je:
\begin{equation}
    \frac{\Delta a_z(s)}{\Delta \delta_V(s)} =\frac{KV}{T^2s^2+2\xi Ts+1}
\end{equation}
\begin{figure}[!ht]
    \centering
    \begin{tikzpicture}[auto, node distance=2cm,>=latex']
        \node[input, name=input](input){};
        \node[block, right of = input, node distance = 3cm] (g1){$\frac{K(T_1s+1)}{s(T^2s^2+2\xi Ts+1)}$};
        \node[block, right of = g1, node distance = 3cm] (g2) {$\frac{T_1s}{1+T_1s}$};
        \node[block, right of = g2] (g3){$\frac{1}{1+T_1s}$};
        \node[anchor = south] (alpha) at ($(g1)!0.6!(g2)$){$\alpha$};
        \node[anchor = south] (theta) at ($(g2)!0.5!(g3)$){$\theta$};
        \node [output, right of = g3] (output) {};
        \node[block, below of = g3] (g4) {$\frac{V}{gT_1}$};
        \node [output, right of = g4] (output2) {};
        \draw [->] (g3) -- node [name=y] {$\Theta$}(output);
        \draw[->] (g4)--node[] {$n_z$}(output2);
        \draw[->] (alpha)|-(g4);
        \draw [draw,->] (input) -- node {$\delta_V$} (g1);
        \draw[->](g1)--(g2);
        \draw[->](g2)--(g3);
\end{tikzpicture}
\caption{Blok dijagram lineariziranog modela longitudinalnog kretanja}
\label{fig:diagLongi}
\end{figure}