Potpun nelinearni dinamički model sastoji iz 12 diferencijalnih jednačina koje su predstavljene ranije. 
Iznimno je teško dobiti analitičko rješenje ovih diferencijalnih jednačina pa se obično pribjegava numeričkoj
simulaciji modela. Zadatak autopilota je da osigura brz prelaz stanja i stabilan odziv u okolini nominalne trajektorije. 
Pokazaće se da se za nominalnu trajektoriju čitav model može raspregnuti što ima za posljedicu 
potpuno razdvajanje modela na dva podsistema. Ova praksa je korištena kod starih letjelica zbog 
uštede računarske moći, ali to danas više nije problem zbog razvoja digitalnih račuara, međutim rasprezanje 
dinamičkog modela je i danas korisno u svrhu sinteze regulatora. Rasprezanje dinamičkog modela 
uvodi netačnosti u model pošto je za rasprezanje potrebno zanemarivanje određenih veličina pa se 
preporučuje ispitivanje regulatora na nelinearnom modelu. U nastavku su sumarno prikazane ranije izvedene 
relacije koje opisuju model projektila krstaste konfiguracije pri čemu treba primjetiti da su ove jednačine 
sada prikazane u koordinatnom sistemu brzine. Korišten je indeks $v$(velocity) da se označi vektor u sistemu brzine
i indeks $b$(body) da se označi vekor u sistemu tijela. Da bi se transformisao vektor iz sistema tijela u sistem brzine 
treba se koristiti inverz matrice transformacije date sa \ref{eq:VtoB}, koji iznosi:
\begin{equation}
    T_b^v = \begin{bmatrix}
            \cos\beta & \sin\beta & 0\\
            -\sin\beta & \cos\beta & 0\\
            0 & 0& 1\\
        \end{bmatrix}
        \begin{bmatrix}
            \cos\alpha & 0 & -\sin\alpha \\
        0& 1& 0\\
        \sin\alpha & 0 & \cos\alpha
        \end{bmatrix}
\end{equation}
Nakon množenja matrica dobija se:
\begin{equation}
    T_b^v\begin{bmatrix}
        \cos\alpha\cos\beta & \sin\beta & -\sin\alpha\cos\beta \\
        -\cos\alpha\sin\beta & \cos\beta & \sin\alpha\sin\beta \\
        \sin\alpha & 0 & \cos\alpha
    \end{bmatrix}
\end{equation}
Sada se konačno mogu napisati svih 12 diferencijalnih jednačina modela u koordinatnom sistemu brzine.
\begin{align}
    &\frac{dV}{dt} = \frac{F_{xv}}{m}\\
    &\frac{d\Psi}{dt} = \frac{F_{yv}}{mv\cos\Theta}\\
    &\frac{d\Theta}{dt} = -\frac{F_{zv}}{mv}\\
    &\frac{dP}{dt}=L/A\\
    &\frac{dQ}{dt}=[M+(I_z-I_x)RP]/I_y\\
    &\frac{dR}{dt}=[N+(I_x-I_y)PQ]/I_z\\
    &\frac{d\psi}{dt}=(R\cos\phi+Q\sin\phi)/\cos\theta\\
    &\frac{d\theta}{dt}=Q\cos\phi-R\sin\phi\\
    &\frac{d\phi}{dt}=P+(R\cos\phi+Q\sin\phi)\tan\theta\\
    &\frac{dx_z}{dt}=V\cos\Theta\cos\Psi\\
    &\frac{dy_z}{dt}=V\cos\Theta\sin\Psi\\
    &\frac{dz_z}{dt}=V\sin\Theta
\end{align}

